\documentclass[10pt]{report}
\usepackage{amsmath, amsthm, amssymb,mathabx}
\usepackage{cmbright}
\usepackage{euler}
\usepackage{setspace}
\usepackage{graphicx}
\usepackage[margin=2.5cm]{geometry}
\usepackage[applemac]{inputenc}
\usepackage[english]{babel}
\usepackage{verbatim}
\usepackage{algpseudocode}

\setlength{\parindent}{0pt}
\onehalfspace

\newcommand \vv{\vvvert}

\begin{document}

\begin{minipage}[t]{0.58\textwidth}
Technische Universit\"at Berlin\\
Institut f\"ur Mathematik\\
Prof. Dr. J\"org Liesen\\
Carlos Echeverr\'ia\\
Luis Garcia Ramos
\end{minipage}
\hfill
\begin{minipage}[t]{0.48\textwidth}
\begin{flushright}
Winter Semester 2014/2015\\
To be submitted in office MA371 on 27.01.2015 before 15.00
\end{flushright}
\end{minipage}
\begin{center}
\textbf{{Numerical Linear Algebra I}}\\
\textbf{Homework 4}
\end{center}
\thispagestyle{empty}
\vspace{0cm}


\begin{enumerate}
    
  \item[\textbf{1.}]Suppose that $A=[a_{ij}]\in\mathbb{C}^{n\times n}$ is  (row) diagonally dominant, i.e. $|a_{ij}|>\sum_{j=1,i\neq j}^n|a_{ij}|$, $i=1,\ldots,n$.
  \begin{enumerate}
    \item[(a)] Show that the Jacobi method applied to $A$ converges for each
    initial approximation $x_0$.\\ 
    Hint: Show that $\|R_J\|_{\infty}<1$. \textit{(7 points)}

    \item[(b)] Show that the Gau\ss-Seidel method applied to $A$ converges for
      each initial approximation $x_0$.\\ 
      Hint: Study the eigenvalues $\lambda$ of the iteration matrix
      $R_G=-(D-L)^{-1}U$ using the equation $Uy=\lambda(D+L)y$. 
      \textit{(7 points)}
  \end{enumerate}

\vspace{0.1cm} 

\item[\textbf{2.}] Consider the matrix $A=\text{tridiag}(-1,2,-1)\in\mathbb{R}^{n\times n}$.
  \begin{enumerate}
    \item[(a)] For  $n=10,11,\ldots,100$, compute $\rho(R_{J/G})$ for the Jacobi and Gau\ss-Seidel methods.\textit{(6 points)}

    \item[(b)] For $n=100$ determine (numerically) an $\omega_*\in(1,2)$ for which the SOR iteration matrix $R_{SOR}(\omega)$ has the smallest spectral radius.\textit{(6 points)}


    \item[(c)] Generate plots of the relative errors $\frac{\|x-x_k\|_{\infty}}{\|x-x_0\|_{\infty}}$ for the Jacobi, Gau\ss-Seidel and SOR (with $\omega_*$ form exercise 2b)) methods applied to $Ax=b$. In order to have an exact solution $x$, compute the right hand side vector as $b=Ax$.  \textit{(6 points)} 

    \item[(c)] Generate plots of the relative errors $\frac{\|x-x_k\|_{\infty}}{\|x-x_0\|_{\infty}}$ for the Jacobi, Gau\ss-Seidel and SOR (with $\omega_*$ from (b)) methods applied to $Ax=b$ (Compute $b=Ax$ in order to have the exact solution $x$).  \textit{(6 points)} 
\end{enumerate}

\vspace{0.1cm} 

\item[\textbf{3.}] Let $S_k,C_k\in\mathbb{C}^{n\times k}$ represent bases of the $k$-dimensional subspaces $\mathcal{S}_k,\mathcal{C}_k$ of $\mathbb{C}^N$. Prove that the following statements are equivalent:
\begin{enumerate}
  \item[(1)] The matrix $C_K^HAS_k\in\mathbb{C}^{k\times k}$ is nonsingular.
  \item[(2)] $\mathbb{C}^{N}=A\mathcal{S}_k\oplus\mathcal{C}_k^\perp$.
\end{enumerate}
\textit{(15 points)}. 
\vspace{0.1cm} 

\item[\textbf{4.}] With the notation established in class, prove that a projection method is well defined at step $k$, if $A$ is nonsingular and $\mathcal{C}_n=A\mathcal{S}_n$. \textit{(8 points)}



%\item[\textbf{3.}] The Jacobi method appears in matrix form as
%  \[x_{k+1}=R_Jx_k+D^{-1}b.\] There is a simple but important modification that can be made to the Jacobi iteration. We  compute the new Jacobi iterates using
%  \[x_k^*=\frac{1}{2}\left(x_k^{j-1}+x_k^{j+1}+h^2b_k\right)\]
%  \begin{enumerate}
%    \item[(a)] Compute $\rho(R_{J/G})$ for the Jacobi and Gau\ss-Seidel methods and $n=10,11,\ldots,100$. \textit{(6 points)}
%
%    \item[(b)] For $n=100$ determine (numerically) an $\omega_*\in(1,2)$ for which the SOR iteration matrix $R_{SOR}(\omega)$ has the smallest spectral radius.
%
%    \item[(c)] Generate plots of the relative errors $\frac{\|x-x_k\|_{\infty}}{\|x-x_0\|_{\infty}}$ for the Jacobi, Gau\ss-Seidel and SOR (with $\omega_*$ form (b)) methods applied to $Ax=b$ (Compute $b=Ax$ in order to have the exact solution $x$).  \textit{(6 points)} 
%    \end{enumerate}

\vspace{3cm} 

\item[\textbf{Note.}] The solutions (.m files) to the programming exercises should be sent by e-mail (1 per group) to the e-mail address: \verb+echeverria@math.tu-berlin.de+ \\

%Remember the following motto by Richard Hamming: 
%\textit{The purpose of computing is insight, not numbers.}

\end{enumerate}

\end{document}

