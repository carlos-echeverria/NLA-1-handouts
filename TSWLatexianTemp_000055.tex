\documentclass[14pt]{report}
\usepackage{amsmath, amsthm, amssymb}
\usepackage{cmbright}
\usepackage{euler}
\usepackage{setspace}
\usepackage[margin=2.5cm]{geometry}
\usepackage[applemac]{inputenc}
\usepackage[english]{babel}
\setlength{\parindent}{0pt}
\onehalfspace

\begin{document}
\textbf{{Review of Linear Algebra: Questions}}\\
\thispagestyle{empty}


\begin{enumerate}
\item Every plane in $\mathbb{R}^n$ is a subspace of $\mathbb{R}^n$.\textbf{
  (Dimension, Linear Independence).}\\ 
\textbf{Solution: False.}  Zero must be an element of a vector space. 

\item  If $A,B$ are two-dimensional subspaces of $R^3$ then their intersection has
dimension at least $1$. 

\textbf{Solution:} \textbf{True}. Take bases of two subspaces, suppose intersection is only $0$, then show
that the four vectors are linearly independent.
$\alpha_1v_1+\alpha_2v_2=\alpha_3w_3+\alpha_4w_4\Rightarrow
\alpha_1v_1+\alpha_2v_2=0, \alpha_4w_4+\alpha_3w_3=0$ One is in $V$ the other
is in $W$ then all coefficients are zero. we have four independent vectors in
$\mathbb{R}^3$. 

\item $det(2AB^{-1})=2det(A)det(B)^{-1}$.\\
\textbf{Solution: False.} The correct answer would be $2^ndet(A)det(B)$, where $n$ is the dimension of the space. 

\item The function $det(A+B)=det(A)+det(B)$\\
\textbf{Solution: False.} Take \[ A= \left[\begin{array}{cc}1 & 0 \\ -1 & 1 \end{array} \right], B= \left[\begin{array}{cc}
1 & 0 \\ 1 & 1 \end{array}. \right]\]
  
\item If $Ax=b$ for $A \in \mathbb{R}^{m\times n}$, $x\in\mathbb{R}^n$, and  
 $b\in\mathbb{R}^m$, then $b$ is a linear combination of the columns of $A$.\\
\textbf{Solution: True.} 

\item There are $4$ linearly independent vectors in $\mathbb{R}^3$.\\ 
\textbf{Solution: False.} 

\item If $A^2=0$ then A is not invertible.\\
\textbf{Solution: True.} $0=det(A^2)=det(A)^2$ implies that $det(A)=0$, so $A$ is not invertible


\item Let $B$ be the $4\times4$ matrix $B=\left[\begin{array}{cccc}
  1&5&7&3\\8&9&0&4\\4&2&1&7\\2&3&5&6\end{array}\right]$ to which we apply the following operations:
        \begin{enumerate}
          \item[(a)] exchange rows 1 \& 4.
          \item[(b)] add row 3 to row 2.
          \item[(c)] exchange rows 1 \& 3.
          \item[(d)] repplace column 1 by column 4.
          \item[(e)] delete row 4.  
        \end{enumerate}
(i) Write the result as a product of 6 matrices.\\
(ii) Write the result as a product $ABC$ (same $B$) of three matrices.

\item if $A=[a_{ij}]\in\mathbb{R}^{3\times 3}$, and $E=\left[\begin{array}{ccc}
  1&2&0\\0&3&0\\0&2&1\end{array}\right]$ then the product $EA$ equals...\\
      (a) the matrix formed by adding twice row 2 to each row of $A$.\\
      (b) the matrix formed by adding twice column 2 to each column of $A$.

\item A set of vectors that are mutually orthogonal can be
  linearly dependent. \\
  \textbf{Solution: False.} 

\item If $A$ and $B$ have rank $3$ then $A+B$ has rank at most $6$.\\
  \textbf{Solution: True.} 

\item The rank $r$ of the $n$ by $n$ matrix with $a_{ij}=i+j$ is $r=n$.\\
\textbf{Solution: False.} Take the 3 by three case. apply elementary operations
and set two rows equal to ones. Rank deficiency. 

\item The inverse of a upper tirangular matrix is upper triangular.\\
  \textbf{Solution: True.} We need to show that $x=A^{-1}e_i\in
  span\{e_i\}$. This is a solution to the linear system
  $Ax=e_i$ where $A$ is upper triangular. By back
  substitution, one can see that the entries below the $i$-th
  entry will be zero. Proof2. $Ae_1\in span\{e_1\}\Rightarrow
  Ae_1=a_1e_1\Rightarrow A^{-1}Ae_1=a_1A^{-1}e_1\Rightarrow A^{-1}e_1\in
  span\{e_1\}$. Now, for $j\leq i$ we need to show the same.
  $Ae_i=\sum_{j=1}^{i}\alpha_je_j$, multiply by inerse....

\item Write down the structure of the Jordan normal form of a matrix $A$.

\item List all the matrix factorizations that you know.

\item Let $A\in\mathbb{R}^{m\times n}$ represent a linear transformation
  $\mathcal{A}:\mathcal{V}\mapsto \mathcal{W}$. What is the dimension of $\mathcal{V}$?
  What is the dimension of $\mathcal{W}$? Write down the rank-nullity theorem.

\item The sum of two eigenvectors is an eigenvector.\\
  \textbf{Solution: False.} 


\item Eigenvectors corresponding to different eigenvalues  are linearly
  independent.\\
    \textbf{Solution: True.} 

\end{enumerate}
\end{document}
