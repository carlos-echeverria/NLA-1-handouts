\documentclass[12pt]{report}
\usepackage{amsmath, amsthm, amssymb}
\usepackage{cmbright}
\usepackage{euler}
\usepackage{setspace}
\usepackage{graphicx}
\usepackage[margin=2.5cm]{geometry}
\usepackage[applemac]{inputenc}
\usepackage[english]{babel}
\usepackage{verbatim}
\setlength{\parindent}{0pt}
\onehalfspace

\begin{document}

\begin{minipage}[t]{0.58\textwidth}
Technische Universit\"at Berlin\\
Institut f\"ur Mathematik\\
Prof. Dr. J\"org Liesen\\
Carlos Echeverr\'ia\\
Luis Garcia Ramos
\end{minipage}
\hfill
\begin{minipage}[t]{0.48\textwidth}
\begin{flushright}
Winter Semester 2014/2015\\
To be submitted in office MA371 on 18.11.2014 before 15.00
\end{flushright}
\end{minipage}
\begin{center}
\textbf{{Numerical Linear Algebra I}}\\
\textbf{Homework 1}
\end{center}

\thispagestyle{empty}

\vspace{0.7cm}

\begin{enumerate}
  \item[\textbf{1.}] Let $R\in\mathbb{C}^{n\times n}$ be upper triangular and normal. Show that $R$ is  diagonal. \textit{(8 points)}


\vspace{0.7cm}



  \item[\textbf{2.}] Let $R_1\in\mathbb{C}^{n\times n}$ and
    $R_2\in\mathbb{C}^{n\times n}$ be upper triangular with identical diagonals
    containing $n$ distinct elements. Suppose that $R_1$ and $R_2$ are 
    unitarily similar, i.e. $U^HR_1U = R_2$ for some unitary matrix $U$.
    Show that $U$ is diagonal. \textit{(10 points)}



\vspace{0.7cm}

  \item[\textbf{3.}] Suppose that $A \in \mathbb{C}^{n \times n}$ has $n$ distinct eigenvalues and
    that $A = U R_1 U^H = V R_2 V^H$, where $U,V$ are unitary and $R_1, R_2$ satisfy the hypothesis of exercise 2. Show that $V^H U$ is unitary and
    diagonal. \textit{(8 points)}


\vspace{0.7cm}

%  \item[\textbf{Exercise 4.}] Find an example of a $2\times2$ or $3\times3$
%    matrix where the converse of Exercise 2 does not hold.

  \item[\textbf{4.}] Given positive real numbers $x,y,z$, define   \[M(x,y,z):=\left[\begin{array}{ccc}0&x&y\\0&0&z\\0&0&0\end{array}\right].\]
    \begin{enumerate}
      \item[(a)] Show that all matrices $M(x,y,z)$ are similar, but that two
         such matrices are orthogonally similar if and only if they are identical. \textit{(4 points)}

      \item[(b)] Compute a formula for the singular values of $M(x,y,z)$ (Hint: Consider the matrix $M(x,y,z)^T M(x,y,z)$). \linebreak\textit{(4 points)}
      \item[(c)] Conclude using (a) that there exist $M_1$ and $M_2$ with
        identical singular values which are not orthogonally similar. \textit{(4 points)} 
      \end{enumerate}
 
\pagebreak

  \item[\textbf{5.}] In this programming exercise you will investigate the
    sensitivity of the Jordan form with respect to perturbations, using MATLAB
    to perform a numerical experiment. In order to see the significant
    digits in the results, set the output format to double precision. This can
    be done by typing the command 'format long' in the console. After that,
    create a script that follows the next steps.

    \begin{enumerate}
      \item[(a)] Set up a $3\times3$ Jordan block $J$ with eigenvalue $i$.
      \item[(b)] Create $50$ diferent $3\times3$ unitary matrices $Q_j$.
      \item[(c)] Apply a similarity transformation ${S}_j=Q^H_jJQ_j$ to the
        Jordan block using each of the orthogonal matrices from (b).
      \item[(d)] Compute a Schur form of the matrices obtained in (c), i.e., $\tilde{S}_{j}=\tilde{Q}^H_j\tilde{J}_j\tilde{Q}_j$. You can use the MATLAB command \verb+[~,B]=schur(A)+.
\end{enumerate}

Are the obtained matrices the same as the original Jordan block? How
    close are they? Answer these questions through the following steps:
    \begin{enumerate}
        \item[(e)] Show in a table the quantity $\|I-Q_j^HQ_j\|_2$ for each of the unitary
          matrices obtained in (b) and give the average order of the error. \textit{(4 points)}
        \item[(f)] Show in a table the quantity $\|iI-\text{diag}(\tilde{J}_j)\|_2$ for the matrices obtained in item (d). Do the computed Schur forms recreate the original eigenvalues? What is the average order of the error of the eigenvalues? \textit{(4 points)}
        \item[(g)] What is the (average) modulus of the (1,3) entry in the computed Schur form? \textit{(4 points)}
  \end{enumerate}

  \end{enumerate}

\end{document}
