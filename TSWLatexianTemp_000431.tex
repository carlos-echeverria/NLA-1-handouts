\documentclass[10pt]{report}
\usepackage{amsmath, amsthm, amssymb,mathabx}
\usepackage{cmbright}
\usepackage{euler}
\usepackage{setspace}
\usepackage{graphicx}
\usepackage[margin=2.5cm]{geometry}
\usepackage[applemac]{inputenc}
\usepackage[english]{babel}
\usepackage{verbatim}
\usepackage{algpseudocode}

\setlength{\parindent}{0pt}
\onehalfspace

\newcommand \vv{\vvvert}

\begin{document}

\begin{minipage}[t]{0.58\textwidth}
Technische Universit\"at Berlin\\
Institut f\"ur Mathematik\\
Prof. Dr. J\"org Liesen\\
Carlos Echeverr\'ia\\
Luis Garc\'ia Ramos
\end{minipage}
\hfill
\begin{minipage}[t]{0.48\textwidth}
\begin{flushright}
Winter Semester 2014/2015\\
To be submitted in office MA371 on 10.02.2015 before 15.00
\end{flushright}
\end{minipage}
\begin{center}
\textbf{{Numerical Linear Algebra I}}\\
\textbf{Homework 5}
\end{center}
\thispagestyle{empty}
\vspace{0cm}


\begin{enumerate}
\item[\textbf{1.}] Let $A \in \mathbb{C}^{n \times n}$ be nonsingular and consider the projection method with $x_k \in x_0 + \mathcal{K}_k(A,r_0)$ and \linebreak $r_k \perp A \mathcal{K}_k(A,r_0)$. Show that \[\|r_{k}\|_2=\|b-Ax_k\|_2=\min_{z \,\in \, x_0+K_k(A,r_0)}\|b-Az\|_2.\]


\vspace{0.1cm} 

\item[\textbf{2.}] For $n \geq 2$, let  $e_1, \ldots, e_n$ be the standard basis vectors of $\mathbb{C}^n$ and consider the matrix 
$A_n=[e_2,e_3,\ldots ,e_n,e_1] \in \mathbb{C}^{n \times n}$. For example, when $n=4$ we have
\[  A_4 = \left[\begin{array}{cccc}
0 & 0 & 0 & 1 \\ 
1 & 0 & 0 & 0 \\
0 & 1 & 0 & 0 \\
0 & 0 & 1 & 0 \\
\end{array} \right].\]
Suppose we solve the system $A_nx=b$ using the projection method from exercise 1 starting with $x_0$=0.
\begin{itemize}
\item[(a)] Show that for every $n \geq 2$ and right hand side $b=e_1 \in \mathbb{C}^n$ we have complete stagnation in the iteration until the last step, i.e., $\|r_0\|_2 = \|r_1\|_2=\cdots =\|r_{n-1}\|_2=1$ and $\|r_n\|_2=0$.
\item[(b)] Show that for every $n \geq 2$ and right hand side $b=[1, 1, \ldots, 1]^T \in \mathbb{C}^n$ we have $\|r_0\|_2=\sqrt{n}$, $\|r_1\|_2=0.$
%\item[(c)]For $n \geq 2$, find all $b \in \mathbb{C}^n$  with $\|b\|_2=1$ such that if the projection method from exercise 1 is used to solve $A_nx=b$ with initial guess $x_0=0$, the iteration converges in $k$ steps, i.e., $\|r_0\|_2 \geq \cdots \geq \|r_{k-1}\|_2>0$ and $\|r_k\|_2=0$.
\end{itemize}  

\item[\textbf{3.}] In this exercise we will compare the properties of different bases of a Krylov subspace computed by various methods.  

\begin{itemize}
\item[(a)]  For each of the four methods described below, write a \verb+MATLAB+ function that takes an arbitrary matrix $A \in \mathbb{C}^{n \times n}$ and a right hand side vector $b\in \mathbb{C}^n$ as input data, and returns a basis $\{v_1, \ldots, v_k\}$ of the Krylov subspace $\mathcal{K}_k(A,b)= \operatorname{span}\{b, Ab, \ldots, A^{k-1}b\}.$ The four methods are:

\begin{itemize}
\item The power method, which computes the basis with $v_1=b/\|b\|$ and $v_{j+1}=Av_j/\|Av_j\|$ for $j=1, \ldots, k-1$. 

\item The Gram-Schmidt method applied to the Krylov sequence $\{b, Ab, \ldots, A^{k-1}b\}.$ 

\item The modified Gram-Schmidt method applied to the Krylov sequence $\{b, Ab, \ldots, A^{k-1}b\}$.
 
\item The Arnoldi method using a Gram-Schmidt step for orthogonalization.

\item The Arnoldi method using a modified Gram-Schmidt step for orthogonalization.
\end{itemize}
\pagebreak
\item[(b)] For the matrix $A= $, $b=$, write a \verb+MATLAB+ script that calls the functions from part (a) to compute four  bases for the Krylov subspace $\mathcal{K}_{100}(A,b)$. Each basis should be stored in a matrix \linebreak $V^{(i)}= [v_1^{(i)}, \ldots v_{100}^{(i)}]$, $i=1, \ldots, 4$. The script should use this data to generate two plots showing  the following quantities:

\begin{itemize}
\item The condition number $\kappa(V_k)$ of the matrices $V_k^{(i)}=[v_1^{(i)}, v_2^{(i)}, \ldots, v_k^{(i)}]$ where $i=1, \ldots, 4$ and $k=1, \ldots 100$.

\item The distance to orthogonality of each matrix  $V_{k}^{(i)}$  in the normalized  Frobenius norm, i.e., the
quantity $\frac{1}{\sqrt{k}}\|I-(V_k^{(i)})^TV_{k}^{(i)}\|_F$ for every  $i=1, \ldots, 4$ and $k=1, \ldots 100$. \end{itemize}



\end{itemize}

\end{enumerate}

\end{document}

