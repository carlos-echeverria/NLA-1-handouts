\documentclass[14pt]{report}
\usepackage{amsmath, amsthm, amssymb}
\usepackage{cmbright}
\usepackage{euler}
\usepackage{setspace}
\usepackage{graphicx}
\usepackage[margin=2.5cm]{geometry}
\usepackage[applemac]{inputenc}
\usepackage[english]{babel}
\usepackage{verbatim}
\setlength{\parindent}{0pt}
\onehalfspace

\begin{document}

\begin{minipage}[t]{0.58\textwidth}
Technische Universit\"at Berlin\\
Institut f\"ur Mathematik\\
Prof. Dr. J\"org Liesen\\
Carlos Echeverr\'ia\\
Luis Garcia Ramos
\end{minipage}
\hfill
\begin{minipage}[t]{0.48\textwidth}
\begin{flushright}
Winter Semester 2014/2015\\
To be submitted in office MA371 on 18.11.2014 before 15.00
\end{flushright}
\end{minipage}
\begin{center}
\textbf{{Numerical Linear Algebra I}}\\
\textbf{Homework 1}
\end{center}

\thispagestyle{empty}

\begin{enumerate}
  \item[\textbf{Exercise 1.}] Let $R\in\mathbb{C}^{n\times n}$ be upper 
    triangular and normal. Show that $R$ must be diagonal. 

  \item[\textbf{Exercise 2.}] Let $R_1\in\mathbb{C}^{n\times n}$ and
    $R_2\in\mathbb{C}^{n\times n}$ be upper triangular with identical diagonals
    containing $n$ distinct elements. Suppose that $R_1$ and $R_2$ are 
    unitarily similar, i.e. $U^HR_1U = R_2$ for some unitary matrix $U$.
    Show that $U$ must be diagonal.

  \item[\textbf{Exercise 3.}] Suppose that $A$ has $n$ distinct eigenvalues and
    that $A = U R_1 U^H = V R_2 V^H$, where $U,V$ are unitary and 
    $R_1, R_2$ satisfy the hypothesis of Exercise B. Show that $V^H U$ is
    diagonal.

%  \item[\textbf{Exercise 4.}] Find an example of a $2\times2$ or $3\times3$
%    matrix where the converse of Exercise 2 does not hold.

  \item[\textbf{Exercise 4.}] Consider the matrix $M(x,y,z)$ for some positive
    real numbers $x,y,z$, where
  \[M(x,y,z)=\left[\begin{array}{ccc}0&x&y\\0&0&z\\0&0&0\end{array}\right].\]
    \begin{enumerate}
      \item[(a)] Show that all matrices $M(x,y,z)$ are similar, but that two
         such matrices are unitarily similar if and only if they are identical.
      \item[(b)] Compute a formula for the singular values of $M(x,y,z)$. 
        (hint: Consider the matrix $M^H M$).
      \item[(c)] Conclude using (a) that there exist $M_1$ and $M_2$ with
        identical singular values which are not unitarily similar.
      \end{enumerate}
  
  \item[\textbf{Programming 1.}] In this exercise you will investigate the
    sensitivity of the Jordan form with respect to perturbations, using MATLAB
    to perform a numerical experiment. In order to see the significant
    digits in the results, set the output format to double precision. This can
    be done by typing the command 'format long' in the console. After that,
    create a script that follows the next steps.
    \begin{enumerate}
      \item Set up a $3\times3$ Jordan block $J$ with eigenvalue $i$.
      \item Create $100$ diferent $3\times3$ orthogonal matrices $Q_j$.
      \item Apply a similarity transformation $\tilde{J}_j=Q^H_jJQ_j$ to the
        Jordan block using each of the orthogonal matrices from (b).
      \item Compute a Schur form of the matrices $[~,S_j]=schur(\tilde{J}_j)$ obtained in (c).
    \end{enumerate}
    Are the obtained matrices the same as the original Jordan block? How
    close are they? Answer these questions through the following steps:
    \begin{enumerate}
        \item Compute the quantity $\|I-Q^TQ\|_2$ for each of the orthogonal
          matrices obtained in (b) and give the average.(Are the orthogonal
          matrices obtained in(b) really orthogonal?)
        \item Compute the quantity $\|iI-diag(S)|$ for each of the , where $\lambda_i$ are the
          eigenvalues of the Schur form.(Does the computed Schur form recreate
          the original eigenvalues?)
        \item What is the modulus of the (1,3) entry in the computed Schur form?
\item Check the orthogonality of the matrices calculated in (b) the by
        computing
      \end{enumerate}

  \end{enumerate}
\end{document}
