\documentclass[14pt]{report}
\usepackage{amsmath, amsthm, amssymb}
\usepackage{cmbright}
\usepackage{euler}
\usepackage{setspace}
\usepackage{graphicx}
\usepackage[margin=2.5cm]{geometry}
\usepackage[applemac]{inputenc}
\usepackage[english]{babel}
\setlength{\parindent}{0pt}
\onehalfspace

\begin{document}

\begin{minipage}[t]{0.58\textwidth}
Technische Universit\"at Berlin\\
Institut f\"ur Mathematik\\
Prof. Dr. J\"org Liesen\\
Carlos Echeverr\'ia\\
Luis Garcia Ramos
\end{minipage}
\hfill
\begin{minipage}[t]{0.48\textwidth}
\begin{flushright}
Winter Semester 2014/2015\\
To be submitted in the lecture on 12.11.2014\
\end{flushright}
\end{minipage}
\begin{center}
\textbf{{Numerical Linear Algebra I}}\\
\textbf{Homework 1}
\end{center}

\thispagestyle{empty}

\begin{enumerate}
  \item[\textbf{Exercise 1:}] Let $R\in\mathbb{C}^{n\times n}$ be upper 
    triangular and normal. Show that $R$ is then diagonal. 

  \item[\textbf{Exercise 2:}] Let $R_1\in\mathbb{C}^{n\times n}$ and
    $R_2\in\mathbb{C}^{n\times n}$ be upper triangular with identical diagonals
    containing $n$ distinct elements. Suppose that $R_1$ and $R_2$ are 
    unitarily similar, i.e. $U^HR_1U = R_2$ for some unitary matrix $U$.
    Show that then $U$ must be diagonal.

  \item[\textbf{Exercise 3:}] Suppose that $A$ has $n$ distinct eigenvalues and
    suppose that $A = U R_1 U^H = V R_2 V^H$, where $U,V$ are unitary and 
    $R_1, R_2$ satisfy the hypothesis of Excercise B. Show that then $V^H U$ is
    diagonal.

  \item[\textbf{Exercise 4:}] Find an example of a $2\times2$ or $3\times3$
    matrix where the converse of Exercise 2 does not hold.

  \item[\textbf{Exercise 5:}] Consider the matrix $M(x,y,z)$ for some positive
    real numbers $x,y,z$.
    \begin{enumerate}
      \item[1] Show that all matrices $M(x,y,z)$ are similar, but that two such
        matrices are unitarily similar if and only if they are identical.
        (There is a general theory in Shapiro's paper, but I think this can be
        worked out be hand for the given $3\times3$ matrix.)
      \item[2] Compute a formula for the singular values of $M(x,y,z)$. 
        (We could give this hint: Consider the matrix $M^H M$).
      \item[3] Conclude using 1. that there exist $M_1$ and $M_2$ with
        identical singular values which are not unitarily similar.
      \end{enumerate}
  
  \item[\textbf{Programming 1:}] We want to investigate the uniqueness
    of the Schur form. We will use the programming language MATLAB to 
    investigate this matter numerically. For accurate numerical results we 
    need to set the output format to double presision. This can be done by
    typing the command 'format long' in the console. After that, create a
    script that follows the next steps.
    \begin{enumerate}
      \item Set up a $3\times3$ Jordan block with eigenvalue $i$.
      \item Create $100$ diferent $3\times3$ orthogonal matrices. (hint: the
        command qr outputs otrhogonal matrices).
      \item Check the orthogonality of the orthogonal matrices.
      \item With each of the orthogonal matrices apply a similarity
        transformation to the Jordan block.
      \item Compute a Schur form of the transformed matrices.
    \end{enumerate}
    Are the obtained matrices the same as the origianl Jordan block? How
    close are they? Equivalently, you can answer this questions by asking: 
    Are the orthogonal matrices obtained from QR-factorization really
    orthogonal? Does the computed Schur form recreate the original eigenvalues?
    What is the modulus of the (1,3) entry in the computed Schur form?

  \end{enumerate}
\end{document}
