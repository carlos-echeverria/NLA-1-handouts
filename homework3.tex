
\documentclass[10pt]{report}
\usepackage{amsmath, amsthm, amssymb,mathabx}
\usepackage{cmbright}
\usepackage{euler}
\usepackage{setspace}
\usepackage{graphicx}
\usepackage[margin=2.5cm]{geometry}
\usepackage[applemac]{inputenc}
\usepackage[english]{babel}
\usepackage{verbatim}
\usepackage{algpseudocode}

\setlength{\parindent}{0pt}
\onehalfspace

\newcommand \vv{\vvvert}

\begin{document}

\begin{minipage}[t]{0.58\textwidth}
Technische Universit\"at Berlin\\
Institut f\"ur Mathematik\\
Prof. Dr. J\"org Liesen\\
Carlos Echeverr\'ia\\
Luis Garcia Ramos
\end{minipage}
\hfill
\begin{minipage}[t]{0.48\textwidth}
\begin{flushright}
Winter Semester 2014/2015\\
To be submitted in office MA371 on 06.01.2015 before 15.00
\end{flushright}
\end{minipage}
\begin{center}
\textbf{{Numerical Linear Algebra I}}\\
\textbf{Homework 3}
\end{center}
\thispagestyle{empty}
\vspace{0cm}



\begin{enumerate}
    
\item[\textbf{1.}] In this problem we will investigate the unicity of the dual vector through a series of small exercises.
  \begin{enumerate}
    \item[(a)] Show that $(\|\cdot\|_1)_D=\|\cdot\|_{\infty}$ and also show that
      $(\|\cdot\|_{\infty})_D=\|\cdot\|_{1}$ without using the duality theorem.

  \item[(b)] For $\hat{x}=\left[\begin{array}{c}1\\\vdots\\1\end{array}\right]$ and $\|\cdot\|_{\infty}$ or $\|\cdot\|_{1}$ find all $y\in\mathbb{C}^{n}$ dual to $\hat{x}$. 

    \item[(c)] Is there an $\hat{x}\in\mathbb{C}^{n}\setminus\{0\}$ having a uniquelly determined dual vector in the infinity norm $\|\cdot\|_{\infty}$?
  \end{enumerate}

\vspace{0.1cm} 

\item[\textbf{2.}] For a given nonsingular matrix $A\in\mathbb{C}^{n\times n}$, Gaussian elimination with partial pivoting produces a factorization $A=PLU$, where $P$ is a permutation matrix, $L$ is unit lower triangular with $|l_{ij}|\leq1$ for all $i,j$, and $U$ is upper triangular.The growth factor is defined as\[\rho(A)=\frac{\max_{i,j}|u_{ij}|}{\max_{i,j}|a_{ij}|}.\]

  \begin{enumerate}
    \item[(a)] It can be shown that $\rho(A)\leq 2^{n-1}$ for each $A$. Show that for $n=2$ this inequality holds. Furthermore, characterize all matrices $A\in\mathbb{C}^{2\times 2}$ for which this is true.
   % \item[(a)] Show that $\rho(A)\leq2^{n-1}$ ofr each $A$.
    \item[(b)] Show that $\rho(A)=2^{n-1}$ for
    \[A_n=\left[\begin{array}{rccc}1&&&1\\-1&1&&1\\\vdots&&\ddots&\vdots\\-1&\cdots&-1&1\end{array}\right]\in\mathbb{C}^{n\times n}\]
    \item[(c)]Compute the following for $n=3,4,...,50$:
      \begin{itemize}
        \item Create a vector $x=\verb+rand+(n,1)$.
        \item Associate a right hand side $b=Ax$ for matrices $A$ as in exercise 2.b.
        \item Calculate the approximate solution $\hat{x}=A\setminus b$ using \verb+MATLAB's+ backslash operator.
        \item Calculate the relative forward error $\alpha=\|\hat{x}-x\|/\|x\|$.
        \item Calculate the relative normwise backward error $\beta=\|r\|/(\|A\|\|\hat{x}\|+\|b\|)$
        \item Compute $\gamma=\frac{\kappa(A)\beta}{\alpha}$
        \end{itemize}
        Create a plot for $\alpha$, $\beta$ and $\gamma$ as a function of $n$. Explain the outcome of these results.
    \end{enumerate}

\vspace{0.1cm} 

\item[\textbf{3.}] Between an adjacent pair of nonzero IEEE single precision real numbers, how many IEEE double precision numbers are there? (Explain and discuss your reasoning). 

\vspace{0.1cm} 

\item[\textbf{4.}] The floating point system $\mathbf{F}$ defined by the elements $x\in\mathbf{F}$ such that \[x=\pm\left(\frac{m}{\beta^t}\right)\beta^e,\] includes many integers but not all of them.
  \begin{enumerate}
    \item[(a)] Give an exact formula for the smallest positive integer $n$ that does not belong to $\mathbf{F}$.
    \item[(b)] In particular, what are the values of $n$ for IEEE single and double precision arithmetic?
    \item[(c)] Figure out a way to verify this result for your own computer. Specifically, design and run a program that produces evidence that $n-3$, $n-2$, and $n-1$ belong to $\mathbf{F}$ but $n$ does not. What about $n+1$, $n+2$, and $n+3$?
    \end{enumerate}


\item[\textbf{Note.}] The solutions (.m files) to the programming exercises should be sent by e-mail (1 per group) to the e-mail address: \verb+echeverria@math.tu-berlin.de+ \\

%Remember the following motto by Richard Hamming: 
%\textit{The purpose of computing is insight, not numbers.}

\end{enumerate}

\end{document}
