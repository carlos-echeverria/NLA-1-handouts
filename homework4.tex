
\documentclass[10pt]{report}
\usepackage{amsmath, amsthm, amssymb,mathabx}
\usepackage{cmbright}
\usepackage{euler}
\usepackage{setspace}
\usepackage{graphicx}
\usepackage[margin=2.5cm]{geometry}
\usepackage[applemac]{inputenc}
\usepackage[english]{babel}
\usepackage{verbatim}
\usepackage{algpseudocode}

\setlength{\parindent}{0pt}
\onehalfspace

\newcommand \vv{\vvvert}

\begin{document}

\begin{minipage}[t]{0.58\textwidth}
Technische Universit\"at Berlin\\
Institut f\"ur Mathematik\\
Prof. Dr. J\"org Liesen\\
Carlos Echeverr\'ia\\
Luis Garcia Ramos
\end{minipage}
\hfill
\begin{minipage}[t]{0.48\textwidth}
\begin{flushright}
Winter Semester 2014/2015\\
To be submitted in office MA371 on 15.01.2015 before 15.00
\end{flushright}
\end{minipage}
\begin{center}
\textbf{{Numerical Linear Algebra I}}\\
\textbf{Homework 4}
\end{center}
\thispagestyle{empty}
\vspace{0cm}


\begin{enumerate}
    
  \item[\textbf{1.}]Suppose that $A=[a_{ij}]\in\mathbb{C}^{n\times n}$ is a (row) diagonally dominant, i.e. $|a_{ij}|>\sum_{j=1,i\neq j}^n|a_{ij}|$, $i=1,\ldots,n$.
  \begin{enumerate}
    \item[(a)] Show that the Jacobi method applied to $A$ converges for each initial approximation $x_0$. Hint: Show that $\|R_J\|_{\infty}<1$. \textit{(6 points)}

    \item[(b)] Show that the Gau\ss-Seidel method applied to $A$ converges for each initial approximation $x_0$. Hint: Study the eigenvalues $\lambda$ of the iteration matrix $R_G=-(D-L)^{-1}U$ using the equation $Uy=\lambda(D+L)y$. \textit{(4 points)}
  \end{enumerate}

\vspace{0.1cm} 

\item[\textbf{2.}] Consider the matrix $A=\text{tridiag}(-1,2,-1)\in\mathbb{R}^{n\times n}$.
  \begin{enumerate}
    \item[(a)] Compute $\rho(R_{J/G})$ for the Jacobi and Gau\ss-Seidel methods and $n=10,11,\ldots,100$. \textit{(6 points)}

    \item[(b)] For $n=100$ determine (numerically) an $\omega_*\in(1,2)$ for which the SOR iteration matrix $R_{SOR}(\omega)$ has the smallest spectral radius.

    \item[(c)] Generate plots of the relative errors $\frac{\|x-x_k\|_{\infty}}{\|x-x_0\|_{\infty}}$ for the Jacobi, Gau\ss-Seidel and SOR (with $\omega_*$ form (b)) methods applied to $Ax=b$ (Compute $b=Ax$ in order to have the exact solution $x$).  \textit{(6 points)} 
    \end{enumerate}

\vspace{0.1cm} 

\item[\textbf{Note.}] The solutions (.m files) to the programming exercises should be sent by e-mail (1 per group) to the e-mail address: \verb+echeverria@math.tu-berlin.de+ \\

%Remember the following motto by Richard Hamming: 
%\textit{The purpose of computing is insight, not numbers.}

\end{enumerate}

\end{document}

